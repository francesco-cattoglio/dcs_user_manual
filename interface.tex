\section{User Interface}

\subsection{Basic concepts}
The central element of the user interface is the node graph editor. Inside it
you create and link nodes of different type to assemble the scene containing all
the computed objects.

There are several different kind of nodes, each with a different functionality
and a different number of inputs, but only one output. This output contains
some data that can be used as an input to one or more nodes, simply by
linking to them. While the same output can be used as input to multiple nodes,
the opposite is an error: you cannot use data coming from 2 outputs into the same input.
The data used in the graph is of one of the following types:

\begin{itemize}
    \item \texttt{Geometry}
    \item \texttt{Vector}
    \item \texttt{Matrix}
    \item \texttt{Time Transform}
    \item \texttt{Interval}
    \item \texttt{Value}
\end{itemize}

One can only connect an output to an input of the same type. For example, it
would not make sense to pass an \textbf{Interval} to a node that requires
a \textbf{Matrix} input.

\subsection{Data types}
\subsubsection{Geometry type}
The most important data type is \textbf{Geometry}. This represents a generic geometric
object, it might be a point, a curve, a surface or a pre-fabricated mesh.
Most of the time we start by creating a geometry of some sort, then
manipulate it with transform functions, and in the end plug the result
into a Renderer node that is the one that turns our data into something
that we visualize on screen.
As we will see later on, parametric transformations and sampling operations
can change the geometric dimension of a Geometry object. As an example,
a given curve can become a surface by applying a parametric transform to it.

\subsubsection{Vector type}
As the name implies, the data contained in this type is just a vector with
user-chosen $x$, $y$ and $z$ coordinates. There is however a very important
difference between a vector and a point: a vector is \textbf{not} a Geometry and
can \textbf{only} be used to build a translation Matrix, as a plane normal or for
visualization purposes. This is because the vector is to be intended as a direction,
not as a generic coordinate in the 3D space. If we write a vector
in homogeneous coordinates, its $w$ component is \textbf{always zero}.

\subsubsection{Matrix type}
This data contains a 4x4 matrix which should be interpreted as a transformation written in
homogeneous coordinates. The matrix can be either parametric or a constant-valued one.
See the nodes section of this chapter to know more about the different matrices that can
be created.

\subsubsection{Time Transform}
This is a very specific 4x4 matrix with an implicit parameter named $t$ as in \textit{time}.
By writing out time-dependent expressions, the user can attach some animation to 
a rendered Geometry object. This can be useful for example to show the movement of a
point along a curve, or to display how a transformation progressively deforms an object
into a final shape.

\subsubsection{Interval and Value}
Both \textbf{Interval} and \textbf{Value} data deals with parameters: the first contains
the parameter and its interval boundaries, while the second contains the parameter and
a specific value in its interval, to be used in a \textbf{Sample parameter} node.

\subsection{Notes}
Assembling a graph does not have a 1:1 correspondency with writing procedural
code (e.g.: classical C code). You are only describing a series of
objects and a set of operations that manipulate those objects. The software
will then compute the correct order in which the operations that you defined
are to be executed to produce the final output.

