\chapter{User Interface}

The user interface consists of a top bar, a status bar in the bottom,
the Scene Viewport and the Node Graph Editor with the Graph Menu Bar.
When the software is started the Node Graph Editor is shown, and it can be
minimized by clicking on a small white triangle on its top left corner;
this will reveal the Scene Viewport behind it. The top bar and the status bar are
always visible.

\section{Node Graph Editor}
\label{section:node_editor}

\begin{figure}[H]
\centering
\includegraphics[width=0.666\textwidth]{figures/graph_editor.png}
%\caption{\label{fig:topbar}}
\end{figure}
The Node Graph Editor is where most of the work is done.
Each node represents an operation that creates or modifies data. Besides the inputs and
the output, many nodes also have internal fields in which the user can write an expression
or choose a value for a given property (e.g. a function, or ``$pi/4$'' for an angle).

The user can perform the following:
\begin{itemize}
    \item To add a node, simply right-click on empty space inside the node graph editor;
            a popup menu will open to display all the available node types, grouped by categories.
            See the following section for a detailed explaination of all menu entries.

    \item To move a node around, left click on its header and drag it to its new position.

    \item You can select a group of nodes by left-clicking on empty space and dragging the mouse,
            or by clicking on nodes while holding the \texttt{ctrl} key, to quickly move all of them.

    \item To remove a node, right-click on its header and select ``delete'' from the dropdown menu.

    \item To connect two nodes, hover over an input or output. When its label turns blue, click it
            and drag with the mouse to the label you want to connect to. 

    \item To remove a connection, double-click it or hover and press the \texttt{delete} key

    \item The pan the graph either right-click on empty space and drag with the mouse,
        or use the mouse scroll (defaults to vertical pan, hold \texttt{shift} for horizontal pan).

    \item The user can zoom in or out by holding the \texttt{ctrl} key and using the mouse scroll.
        Clicking the mouse scroll (a.k.a. \texttt{Mouse3 button}) will reset the zoom level.

\end{itemize}

\begin{figure}[H]
\centering
\includegraphics[width=0.4\textwidth]{figures/right_click_popup.png}
\caption{\label{fig:rightclick_manu} The right-click menu, with the ``Geometry'' submenu open}
\end{figure}

\subsection{Geometry submenu}
The geometry submenu contains the following nodes:

\subsubsection{Curve}

\hspace{\baselineskip}
\begin{tabular}{g|l}
    \hline
    Inputs: & Interval              \\
    \hline
    Fields: & fx, fy, fz            \\
    \hline
    Output: & 1D Geometry object  \\
    \hline
\end{tabular}
\vspace{5pt}

This node is used to create a parametric curve from the input parameter Interval
and the given expressions for fx, fy and fz. Simple to use,
just make sure that the expressions are using the correct parameter name
(i.e. the same name used by the input Interval)

\subsubsection{Bezier}

\hspace{\baselineskip}
\begin{tabular}{g|l}
    \hline
    Inputs: & 2, 3 or 4 Points (0D Geometry objects)\\
    \hline
    Fields: & Quality\\
    \hline
    Output: & 1D Geometry object\\
    \hline
\end{tabular}
\vspace{5pt}

This node uses its inputs as control points for a Bezi\`er curve. The degree of the
curve depends on how many inputs have been connected (empty inputs are ignored).
The Quality field determines the quality of the numerical approximation
of the ideal Bezi\`er curve.

\subsubsection{Surface}

\hspace{\baselineskip}
\begin{tabular}{g|l}
    \hline
    Inputs: & 2 Intervals\\
    \hline
    Fields: & fx, fy, fz\\
    \hline
    Output: & 2D Geometry object\\
    \hline
\end{tabular}
\vspace{5pt}

Similarly to the curve one, this node can be used to create a parametric surface
from two Intervals and the given expressions for fx, fy and fz.
Both intervals are required and the user should use both parameters:
if one parameter is never used in the expressions, no visual output
will be produced, since the final result will be a degenerate surface.

\subsubsection{Plane}

\hspace{\baselineskip}
\begin{tabular}{g|l}
    \hline
    Inputs: & 1 point (0D Geometry), 1 Vector\\
    \hline
    Fields: & none\\
    \hline
    Output: & Mesh Geometry object\\
    \hline
\end{tabular}
\vspace{5pt}

This node takes one point and a normal vector to create a special representation
of a plane. In particular, the representation consists of an evely-spaced grid centered
about the point given as input.

Please note the output is \textbf{not} a 2D Geometry, but a Mesh Geometry;
this means that you will not be allowed to apply a parametric transformation
or a sample operation to a it, only a constant-valued transformation can be applied.
This is because the purpose of this node is to have something useful for
visualization purposes, not to have a built-in parametric plane.

\subsubsection{Primitive}

\hspace{\baselineskip}
\begin{tabular}{g|l}
    \hline
    Inputs: & none\\
    \hline
    Fields: & Kind (a drop-down menu), Size\\
    \hline
    Output: & Mesh Geometry object\\
    \hline
\end{tabular}
\vspace{5pt}

This node allows the user to create a basic primitive of the given Kind
(Cube, Sphere, Cylinder, Cone, Pyramid or Dice). The Size fields allows
adjusting the primitive dimensions. See the question mark next to the 
slider for a description on how this value effects the output.

Just like the Plane node, the output is a Mesh Geometry, which is useful
for quick visualization or similar purposes. Parametric transformations
on these objects are \textbf{not} allowed.

\subsection{Parameters submenu}
This submenu contains all the nodes related to parameters

\subsubsection{Interval}

\hspace{\baselineskip}
\begin{tabular}{g|l}
    \hline
    Inputs: & none\\
    \hline
    Fields: & Name, Begin, End, Quality\\
    \hline
    Output: & Interval\\
    \hline
\end{tabular}
\vspace{5pt}

This node defines a parameter. By assigning a name, you will then be able
to use this parameter in expressions for parametric curves, surfaces and matrices.
Begin and End define the closed range in which the parameter lives, while Quality
allows the user to choose how many discretization points will be used when creating
a curve or a surface out of this parameter.

\subsubsection{Value}

\hspace{\baselineskip}
\begin{tabular}{g|l}
    \hline
    Inputs: & none\\
    \hline
    Fields: & Name, Value\\
    \hline
    Output: & Parameter Value\\
    \hline
\end{tabular}
\vspace{5pt}

In a similar way to the Interval node, the Value node lets you define a parameter
and assign a specific value to it. You can use it either as a ``variable''
to be used as input for a Matrix (e.g. define \textit{theta} and then use it as the
angle inside the matrix expressions) or as an input to the Sample Parameter node.

\subsubsection{Sample Parameter}

\hspace{\baselineskip}
\begin{tabular}{g|l}
    \hline
    Inputs: & 1D or 2D Geometry object, Parameter Value\\
    \hline
    Fields: & none\\
    \hline
    Output: & 0D or 1D Geometry object\\
    \hline
\end{tabular}
\vspace{5pt}

The parameter sampling operation allows you to ``downgrade'' the dimension of a parametric
Geometry object, by fixing a parameter to the value given as input. Mostly useful for
obtaining $u$ or $v$ curves.

In order for the operation to succeed, the Geometry must have a parameter with the same name
as to the one used in the Value input (e.g. if a surface is $S(p, q)$, the input value cannot
be named $r$, only $p$ or $q$ will be accepted). The value must also be inside
the parameter's range.

Please note that 1D Geometry created with the Bezi\`er
curve uses a hidden parameter name, and cannot therefore be sampled.

\subsection{Transformations submenu}
This submenu contains all nodes used to define transformation matrices and to
apply them to Geometry objects.

\subsubsection{Generic Matrix}

\hspace{\baselineskip}
\begin{tabular}{g|l}
    \hline
    Inputs: & Parameter (either Interval or Value, optional)\\
    \hline
    Fields: & matrix elements\\
    \hline
    Output: & Matrix object\\
    \hline
\end{tabular}
\vspace{5pt}

This node is used to define a generic Matrix, by writing one expression for each
matrix element. If an Interval is given as an input, then the output matrix will be
a parametric one.
Note that due to the nature of the software, there is no way to define a projection matrix,
since the last row of the matrix is fixed to $[0, 0, 0, 1]$ and cannot be modified.

As usual, if you provide a parameter as an input, make sure you
will be actually using it inside the expressions to prevent any kind of visualization
issue.

\subsubsection{Rotation Matrix}

\hspace{\baselineskip}
\begin{tabular}{g|l}
    \hline
    Inputs: & none\\
    \hline
    Fields: & Axis (drop down menu), Angle\\
    \hline
    Output: & Matrix object\\
    \hline
\end{tabular}
\vspace{5pt}

This node allows the user to quickly define a matrix which represents a rotation
of given Angle (in radians) around the chosen Axis.

\subsubsection{Translation Matrix}

\hspace{\baselineskip}
\begin{tabular}{g|l}
    \hline
    Inputs: & Vector\\
    \hline
    Fields: & none\\
    \hline
    Output: & Matrix Object\\
    \hline
\end{tabular}
\vspace{5pt}

Similarly to the previous one, this node allows the user to quickly define a translation
matrix given the input Vector.

\subsubsection{Transform}

\hspace{\baselineskip}
\begin{tabular}{g|l}
    \hline
    Inputs: & Geometry object, Matrix object\\
    \hline
    Fields: & none\\
    \hline
    Output: & Geometry object\\
    \hline
\end{tabular}
\vspace{5pt}

This node takes a matrix and applies it to a Geometry, producing a transformed Geometry.
If the input matrix was a parametric matrix, then a parametric transformation will be applied.

Please note that a parametric transformation is not ``blindly'' applied to an object; instead
the parameter names will be read and the operation will be performed accordingly.
As an example, consider a circle on the $xy$ plane parametrized in $s$: if we apply
a parametric translation with value $[0, 0, s]$, the output will be a circular helix.

\subsubsection{Time Transform}

\hspace{\baselineskip}
\begin{tabular}{g|l}
    \hline
    Inputs: & none\\
    \hline
    Fields: & matrix elements\\
    \hline
    Output: & Time Transform object\\
    \hline
\end{tabular}
\vspace{5pt}

This node creates a Time Transform object, read about Data Types in section \ref{section:data_types}
for more informations. All the matrix element expressions may contain the parameter $t$, as in ``time''.

\subsection{Point}

\hspace{\baselineskip}
\begin{tabular}{g|l}
    \hline
    Inputs: & none\\
    \hline
    Fields: & $x$, $y$, $z$\\
    \hline
    Output: & 0D Geometry object\\
    \hline
\end{tabular}
\vspace{5pt}

A very simple node to create a point. The implicit $w$ coordinate is set to $1$

\subsection{Vector}

\hspace{\baselineskip}
\begin{tabular}{g|l}
    \hline
    Inputs: & none\\
    \hline
    Fields: & $x$, $y$, $z$\\
    \hline
    Output: & Vector object\\
    \hline
\end{tabular}
\vspace{5pt}

A very simple node to create a Vector. As described in the Data Types section, a Vector
is to be interpreted as a direction, and the implicit $w$ coordinate is set to $0$.

\subsection{Geometry Renderer}


\hspace{\baselineskip}
\begin{tabular}{g|l}
    \hline
    Inputs: & Geometry, Time Transform (optional)\\
    \hline
    Fields: & Thickness, Color\\
    \hline
    Output: & none\\
    \hline
\end{tabular}
\vspace{5pt}

This node is the one responsible for taking a Geometry object and rendering it to the screen.
Every frame the the Time Transform is applied to it by evaluating the
transformation with the current value of $t$. The user can choose what Color to use for any
Geometry, while the Thickness value only affects the display of 1D geometries (curves).

\subsection{Vector Renderer}

\hspace{\baselineskip}
\begin{tabular}{g|l}
    \hline
    Inputs: & Point, Vector\\
    \hline
    Fields: & Thickness, Color\\
    \hline
    Output: & none\\
    \hline
\end{tabular}
\vspace{5pt}


This node is used to display a Vector, by representing it as an arrow.
Since a Vector is only a direction, the user must also provide the application Point (i.e: the "tail") of the vector.
Just like with the Geometry Renderer, one can choose a color and the thickness of the arrow.

\section{Top Bar}

\begin{figure}[H]
\centering
\includegraphics[width=0.8\textwidth]{figures/top_bar.png}
%\caption{\label{fig:topbar} The right-click menu, with the ``Geometry'' submenu open}
\end{figure}

As it was explained before, any Geometry Renderer can have a Time Transform attached to it.
This transformation contains a parameter named $t$ that changes in real time while the
user interacts with the rendered scene, as opposed to all other parameters that are evaluated
only once when the scene is generated.
The Top Bar gives the user control over the value of this parameter $t$. It contains the following:

\begin{itemize}
    \item Start: the expression for the beginning of the parameter interval. To use
        the constant $\pi$, write it as \texttt{pi}.
    \item End: the expression for the end of the parameter interval.
    \item Play/Pause: by clicking ``Play'' the parameter $t$ will increment as time passes,
        while by clicking ``Pause'' the parameter will stay frozen to the value selected by the user.
    \item nameless slider: the user can click and drag the slider to manually assign a value to $t$
    \item Loop checkbox: as the name implies, if this box is ticked the animation will loop
        (i.e. every time $t$ reaches the end of the interval, it will be rewinded to the start)
    \item Speed: defines how fast $t$ advances when ``Play'' is clicked. Defaults to $0.75$,
        meaning that 1 second of wall clock time will advance $t$ by $0.75$. The user can either
        write down the speed or use the \texttt{+/-} signs to change it by small steps.
        Negative speeds are allowed!
\end{itemize}


\section{Graph Menu Bar}

\begin{figure}[H]
\centering
\includegraphics[width=0.5\textwidth]{figures/graph_menu_bar.png}
%\caption{\label{fig:topbar} The right-click menu, with the ``Geometry'' submenu open}
\end{figure}

The Graph Menu Bar contains 4 buttons, which function is easy to guess by their labels:

\begin{itemize}
    \item Clear Graph: deletes all nodes and connections.
    \item Open File: clears the graph and loads a previously saved file.
    \item Save File: saves the current graph to a file.
    \item Generate Scene: when the user is done modifying the graph, clicking this
        will trigger the generation of the scene.
\end{itemize}

Clearing the graph, opening and saving a file will display a dialog asking to confirm the operation.
Please note that when saving a file the extension \texttt{.toml} will be automatically added if
the name does not contain it. If any name conflict arises, then the user will be asked whether
the software should overwrite the file or abort the operation.

\section{Scene Viewport}

After the scene is generated, the Graph Editor is minimized and the scene containing the generated
geometries is shown. Navigation is very easy: left-click and drag to rotate the view and use
the mouse scroll or numpad's ``+/-'' to control zoom. Currently it is not possible to pan around the scene.
The scene contains some colored lights placed in specific points in the attempt to make it easier to
understand what shapes the user is looking at. However this also means that some combinations of shape
and colors might produce dark spots in some geometries, try using a different color if this happens.

\section{Status Bar}

\begin{figure}[H]
\centering
\includegraphics[width=0.6\textwidth]{figures/status_bar.png}
\caption{\label{fig:status_bar} The status bar reporting an error in the user-created graph}
\end{figure}

The status bar is an often overlooked part of the User Interface, probably due to its location
at the very bottom of the window. It is however extremely useful
because when the software detects a mistake in the graph, it will notify the user
with a (hopefully) useful message to help understand where the error is.
It will also report if scene generation and file operations were successful or not.


