\chapter{Basic Concepts}

The central element of the user interface is the node graph editor. Inside it
you create and link nodes of different type to assemble the scene containing all
the computed objects.

There are several different kind of nodes, each with a different functionality
and a different number of inputs, but only one output. This output contains
some data that can be used as an input to one or more other nodes, simply by dragging a link
between them using the mouse. While the same output can be used as input to multiple nodes,
the opposite is an error: you cannot feed data coming from 2 outputs into one input.

One can only connect an output to an input of the same type. For example, it
would not make sense to pass an \textbf{Interval} to a node that requires
a \textbf{Matrix} input.

\section{Data types}
\label{section:data_types}
Before we dive in the user interface, it is important to briefly talk about
all the different Data types that the user will encounter while using the software.

\subsection{Geometry}
The most important data type is the \textbf{Geometry}. This represents a generic
object, it might be a point (0D), a curve (1D), a surface (2D) or a pre-fabricated mesh.
Most of the time we start by creating a geometry of some sort, then
manipulate it with transform functions, and in the end plug the modified geometry
into a Renderer node to visualize the result on screen.
As we will see later on, parametric transformations and sampling operations
can change the number of dimension of a Geometry object. As an example,
applying a parametric transform to a given curve may turn it into a surface.

\subsection{Vector}
As the name implies, the data contained in this type is just a vector with
user-chosen $x$, $y$ and $z$ coordinates. There is however a very important
difference between a vector and a point: a vector is \textbf{not} a Geometry and
can \textbf{only} be used to build a translation Matrix, as a plane normal or for
visualization purposes. This is because the vector is to be intended as a direction,
not as a generic point in the 3D space. If we were to write such vector
in homogeneous coordinates, its $w$ component is \textbf{zero}.

\subsection{Matrix}
This data contains a 4x4 matrix which should be interpreted as a transformation written in
homogeneous coordinates. The matrix can be either parametric or a constant-valued one.
See section \ref{section:node_editor} to know more about the different matrices that can
be created.

\subsection{Time Transform}
This is a very specific 4x4 matrix containing a parameter named $t$ as in \textit{time}.
By writing time-dependent expressions, the user can attach an animation to 
a rendered Geometry object. This can be useful to show the movement of a
point along a curve, or to display how a transformation progressively deforms an object
into a final shape.

\subsection{Interval and Value}
Both \textbf{Interval} and \textbf{Value} data are related to parameters.
An \textbf{Interval} defines a name for a parameter and the interval on which it lives.
Please note that this is always to be interpreted as a \textbf{closed} interval.
A \textbf{Value} on the other hand contains the parameter name and a specific value that
it can assume; this information can be used in a \textbf{Sample parameter} node.

\section{Notes}
Assembling a graph does not have a 1:1 correspondency with writing procedural
code (e.g.: typical \CC{} code). You are only describing a series of
objects and a set of operations that manipulate those objects, but you do not have to
to define the order in which the computations will be executed. The order of execution
will be decided by the software by looking at how nodes depend from each other.

Another important aspect of the node programming approach is that all the operations
start from some \textit{sources} (i.e: geometries and parameters) and \textit{flow} to the graph
to reach a \textit{sink} (i.e: visualization). It is an error to create a loop in the graph because
it would break this flow and there would be no clear way on how to run all the
needed computations.
As an example, a loop might be created by using the output of a
transform as its own input.

