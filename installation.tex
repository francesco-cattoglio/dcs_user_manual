\chapter{Installation}
The software does not require any installation, you can simply download it
and unzip in a folder of your choice. Keep in mind that both Windows and OS X
block executables downloaded from the internet, therefore you will need to
right-click on the executable and chose the "Open" option. This will ask if
the software is safe to execute, and this confirmation only needs to be given once.
You can skip the rest of this chapter if you have no need of compiling from sources.

\section{Compiling from sources}

If you have access to the sources you can compile them under Linux and OS X
quite easily, since we provide with some scripts to setup everything
in a mostly automated way.

Under Windows, the code has been developed and compiled under MSYS2, which
is a software distro and building platform for Windows. What that means is
you have a linux-like shell environment and package manager, with a
\texttt{mingw64}-based compiler toolchain. The generated executables will run
on any 64-bit Windows machine, provided you bundle a few \texttt{dll} files.

Under OS X, the code can been compiled using Apple's version of Clang (the default system
compiler). Some additional tools like CMake are however required, the developer
suggests the use of the homebrew package manager for fetching all the dependencies.

\subsection{Building instructions}
Both the software and its dependencies are written in \CC{} and use CMake as a
build system. You can simply \texttt{git clone} all the dependencies, cmake install
them and be good to go.

However, if you want a more guided way of building everything from scratch,
you can start by cloning a repository that contains a series of scripts to help
you in the process. Those will download and build
all the parts with the needed CMake flags, using a temporary build folder
so that your system root does not get polluted with any CMake install used
in the process.

For compiling you will need the following tools:
\begin{itemize}
    \item a recent compiler with \CC{14} support
    \item \texttt{cmake}
    \item the \texttt{ninja-build} build system
    \item \texttt{pkg-config}
\end{itemize}

You will also need some development libraries. For Debian-based linux, the list of packages is as follows:
\begin{itemize}
    \item \texttt{libglfw3-dev}
    \begin{itemize}
        \item  or, as an alternative, all of the following: \\
            \verb|libx11-dev libxrandr-dev libxinerama-dev libxcursor-dev libxi-dev|
    \end{itemize}
    \item \texttt{libgl1-mesa-dev}
    \item \texttt{libgtk-3-dev}
\end{itemize}
Please note that the dependency names might be slightly different depending on your distribution.

To guided compilations via provided scripts, the list of operations is as follows:
\begin{itemize}
    \item \texttt{git clone http://github.com/francesco-cattoglio/cmake\_scripts.git}
    \item run the \texttt{./cmake\_scripts/all\_git\_downloads.sh} script
    \item \texttt{mkdir tmp\_folder}
    \item export environment var: \texttt{export CMAKE\_BUILDS\_PREFIX=/full/path/to/tmp\_folder}
    \item launch the \texttt{./cmake\_scripts/build\_everything.sh} script
    \begin{itemize}
        \item this will trigger the individual build scripts for each dependency
    \end{itemize}
    \item launch the final executable in the \texttt{./dcs/build} folder
\end{itemize}
Since the build script makes use of relative paths, please \textbf{do not change folder} before launching it.

If something goes wrong, instead of calling the \texttt{build\_everything} script,
you can build and install dependencies one at a time and keep the build
scripts as references for the needed CMake flags.

\subsection{Deployment}
The procedure to deploy the executable is slightly different depending
on the target OS. If you used the provided scripts, both the Linux and OS X builds
should be standalone executables with no external dependencies. The application might be
named ``dcs'' instead of ``franzplot'', depending on the version of the sources you are using.

On Windows, to help mitigate some bugs in the OpenGL Intel driver that is used on most
laptops with integrated Intel HD graphics, the build process uses SDL2 and Google's
ANGLE software. When distributing your executable, you have to also bundle it
with the libraries provided in the \texttt{win\_deploy} subfolder:
\texttt{libglesv2.dll}, \texttt{libegl.dll}, \texttt{libwinpthread-1.dll}, \texttt{sdl2.dll}.
It is still possible to distribute a single \texttt{.exe} file by using a packager.
The tool that we used for such a task is called NSIS: Nullsoft Scriptable Install System,
which is freely available under an open source license.

To create a standalone launcher, follow those steps:
\begin{itemize}
    \item install the NSIS software (\texttt{https://nsis.sourceforge.io/Main\_Page})
    \item compile the software as per the previous section
    \item move the compiled executable inside the \texttt{win\_deploy} folder;
        \textbf{make sure to rename it to franzplot.exe}
    \item right click on the \texttt{create\_launcher.nsi} file and chose
        ``compile script'' from the menu
    \item check that the created launcher works properly by moving it to a different
        folder and executing it
\end{itemize}
