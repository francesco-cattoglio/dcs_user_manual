\section{Installation}
The software does not require any installation, you can simply download it
and unzip in a folder of your choice. 

\section{Compiling from sources}

If you have access to the sources you can compile them under Linux and OS X
quite easily, since we provide with some scripts to setup everything
in a mostly automated way.

Under Windows, the code has been developed and compiled under MSYS2, which
is a software distro and building platform for Windows. What that means is
you have a linux-like shell environment and package manager, with a
mingw64-based compiler toolchain. The generated executables will run
on any 64-bit Windows machine, provided you bundle a few \texttt{dll} files.

\subsection{Building instructions}
Both the software and its dependencies are written in \CC and use CMAKE as a
build system. You can simply [git clone] all the dependencies, cmake install
them and be good to go.
However, if you want a more guided way of building everything from scratch,
you can start by cloning a repository that contains a series of scripts to help
you in the process: [git cmake\_scripts]. Those will download and build
all the parts with the needed cmake flags, using a temporary build folder
so that your system root does not get polluted with any cmake install used
in the process.

For compiling you will need the following tools:
\begin{itemize}
    \item a recent compiler with \CC{14} support
    \item \texttt{cmake}
    \item the \texttt{ninja-build} build system
    \item \texttt{pkg-config}
\end{itemize}

You will also need some development libraries. For Debian-based linux, the list of packages is as follows:
\begin{itemize}
    \item \texttt{libglfw3-dev}
    \begin{itemize}
        \item  or, as an alternative, all of the following: \\
            \verb|libx11-dev libxrandr-dev libxinerama-dev libxcursor-dev libxi-dev|
    \end{itemize}
    \item \texttt{libgl1-mesa-dev}
    \item \texttt{libgtk-3-dev}
\end{itemize}
Please note that the dependency names might be slightly different depending on your distribution.

The list of operations is as follows:
\begin{itemize}
    \item \texttt{git clone http://github.com/francesco\_cattoglio/cmake\_scripts.git}
    \item run the \texttt{./cmake\_scripts/all\_git\_downloads.sh} script
    \item \texttt{mkdir tmp\_folder}
    \item export environment var: \texttt{export CMAKE\_BUILDS\_PREFIX=/full/path/to/tmp\_folder}
    \item launch the \texttt{./cmake\_scripts/build\_everything.sh} script
    \begin{itemize}
        \item this will trigger the individual build scripts for each dependency
    \end{itemize}
    \item launch the final executable in the \texttt{./dcs/build} folder
\end{itemize}
Since the build script makes use of relative paths, please \textbf{do not change folder} before launching it.

If something goes wrong, instead of calling the \texttt{build\_everything} script,
you can build and install dependencies one at a time and keep the build
scripts as references for the needed cmake flags.

\subsection{Deployment}
To deploy the executable, the procedure is a bit different, depending
on which OS you are on. For Linux, everything is really simple since
all of the libraries are compiled in as static libs.

On Windows, ...

On OS X, ...
